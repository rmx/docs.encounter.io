\chapter{Overview}

\section{Basic concepts}

\begin{description}
	\item [Effect] A small, atomic game event.\\
		\textit{E.g., a unit attacking another unit.}
	\item [Modifier] Modifies unit attributes or game events.\\
		\textit{E.g., reduce all incoming damage, or make a unit move faster.}
	\item [Reaction] Tiggers effects after a given game event.\\
		\textit{E.g., Heal a unit after it has taken damage.}
	\item [Condition] A condition that can be attached to an effect, a modifier or a reaction. The parent is then only active if the condition is met.\\
		\textit{E.g., make a damage reduction modifier only apply for fire damage, or make a spell only stun the target if it is a demon.}
	\item [Action] An action performed by a unit. Can contain modifiers, reactions, effects.\\
		\textit{E.g., a spell with an increased crit chance (modifier) that deals damage to the target (effect), and restores half the cost if the spell missed (reaction).}
	\item [Aura] A named compound state attached to a unit. Can contain modifiers, reactions, effects.\\
		\textit{E.g., an aura that periodically deals frost damage (periodic effect), slows the movement speed (modifier), and freezes the target after it takes additional damage (reaction).}
\end{description}


\begin{tikzpicture}[
    ->,
    >=stealth',
    concept/.style={draw, rounded corners, thick, rectangle split, rectangle split parts=2, text ragged, text width=8em, minimum height=2cm},
    node distance=1.5cm
    ]

    % Action
    \node[concept] (ACTION) 
    {
        \textbf{Action}
        \nodepart{second}
    };

    % Reaction
    \node[concept,below right=of ACTION] (REACTION) 
    {
        \textbf{Reaction}
        \nodepart{second}
    };

    % Aura
    \node[concept,above right=of REACTION] (AURA) 
    {
        \textbf{Aura}
        \nodepart{second}
    };

    % Effect
    \node[concept,below=of REACTION] (EFFECT) 
    {
        \textbf{Effect}
        \nodepart{second}
    };

    % Modifier
    \node[concept,below=of EFFECT] (MODIFIER) 
    {
        \textbf{Modifier}
        \nodepart{second}
    };
    
    % Condition
    \node[concept,right=of EFFECT] (CONDITION) 
    {
        \textbf{Condition}
        \nodepart{second}
    };

    % draw the paths
    \draw (ACTION) edge (REACTION);
    \draw (AURA) edge (REACTION);
    
    \draw (ACTION) edge[bend right=20] (EFFECT);
    \draw (REACTION) edge (EFFECT);
    \draw (AURA) edge[bend left=20] (EFFECT);

    \draw (ACTION) edge[bend right=20] (MODIFIER);
    \draw (EFFECT) edge (MODIFIER);
    \draw (AURA) edge[bend left=20] (MODIFIER);

    \draw (EFFECT) edge (CONDITION);
    \draw (REACTION) edge (CONDITION);
    \draw (MODIFIER) edge (CONDITION);

\end{tikzpicture}

